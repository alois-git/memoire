

  In conclusion, results of the tests show that providing a robust and fast internet connection to compagnies and individuals using MPTCP coupled with OpenVPN is something possible.

  The setup is minimal and not too expensive. The elements needed are two or more internet connections, a router running a MPTCP compatible kernel for the OpenVPN client and
  a server running a MPTCP compatible kernel for the OpenVPN server having a high bandwith.

  The basic configuration is simple but can be more complicated if the goal is to achieve good performances and try to maximize the speed of the internet connection.

  In term of performance, the hardware need to be powerfull enough otherwise the throughput will be limited by OpenVPN and its intensive
  use of the CPU to encapsulate and decapsulate packets. Here with the TP-LINK N900 router the maximum throughput it could reach was 57mbit/s before the CPU was saturated, with
  nowdays internet connection going up to 100mbit/s, this limit is not very high.

  In this paper we saw that using OpenVPN with MPTCP on links having low bandwith delay product gives good performance.
  However using links with high bandwith delay product decrease the performances.
  In some cases the use of MPTCP is not even worth it, UDP as the OpenVPN protocol would be a better choice in term of performances if we ommit the fact that it is less robust against failure.

  I have investigated and tried to solve the problem concerning the high bandwith delay product links with some level of success by adjusting the
  TCP windows size on both OpenVPN end side and trying different congestion algorithms.
  The performances were increased in every cases but depending on the configuration it was more or less significant.

  This being said, it will require internet connections to have a low RTT between the OpenVPN client and the OpenVPN server to get an acceptable user experience.
  Normal VDSL or ADSL connections should be adequate but mobile internet connection (3G) or sattelite internet connection will not be suitable for this kind of setup.

  Finally all the tests in this paper were done in a closed lab environnement. An interesting thing to do would be to try that in the real world environnement over the internet, to determine
  if additional problems than the ones represented in these experimental tests are encountered, potentially making the use of MPTCP coupled with OpenVPN not a viable solution.
  Another interesting thing would be to try using a different technology than OpenVPN and test if it would give better performances.
